% !TEX root = C:/Users/piyus/knowledge/Project_Specific_Knowledge/public/fm_radio/stages/variable_gain_amp/variable_gain_amp.tex
\documentclass[12pt, letterpaper]{article}

\usepackage{hyperref}
\usepackage{graphicx}
\graphicspath{ {C:/Users/piyus/knowledge/Project_Specific_Knowledge/public/fm_radio/stages/variable_gain_amp/pictures} }

\title{Variable Gain Amplifier Notes}
\author{Piyush Sud}
\date{10/13/2024}
\begin{document}
\maketitle

\pagebreak

\section{High Level Design}

\begin{itemize}
    \item What range of amplitudes are we looking for?
    \item Based on this: \url{https://forum.allaboutcircuits.com/threads/voltage-on-an-antenna.88982/}
    \item It seems like FM receiver gain is generally on the order of 100 - 120 dB and the input is typically in the low uV to low mV range.
    \item The power gain of the input stage is -3 dB, HF amp is 31.5 dB, mixer is 2.4 dB, fm detector is 0 dB, and output headphone amplifier is 20 dB. This means that the overall gain is 50.9 dB.
    \item This seems like a good variable gain amplifier: https://www.digikey.com/en/products/detail/texas-instruments/VCA824IDGST/1766828?s=N4IgTCBcDaIGoGECCAOMAWAkgEQOIGUAVEAXQF8g
    \item Bandwidth is 320 MHz, which is much higher than what we need.
    \item The gain can be adjusted from 2 to 32 dB.
    \item With 2 variable gain amplifiers in series, the overall gain of the system ranges from 54.9 dB to 114.9 dB. This accomodates a wide range of signal strengths.
\end{itemize}

\section{Detailed Design}

\begin{itemize}
    \item The maximum gain is set by the resistors RF and RG.
    \item For some reason (probably because it creates a pole, reducing the BW), they used small resistor values for Rf and Rg in their example circuits, so let's do the same. Let's choose Rf = 500 ohms. Since we want the gain to be able to up to 32 dB (40 V/V), then we need Rg to go down to 12.5 ohms. If we want the gain to down to 3 dB (2V/V), then we need Rg = 250. Therefore the pot range we are looking for is [12.5, 250]. The bottom value in the range should be easy to achieve since most pots can typically go down to 1 to 2 percent of their max value. 
    \item It seems like a resistor between the feedback pin and the input is only used if we need the output voltage to be both a function of Vg and Vin. However, in this case, we only really need it to be a function of Vin since we can set Rf/Rg using a potentiometer, and the output voltage is also proportional to Rf/Rg.
    \item The input impedance is high impedance, so we can just put a 50 ohm resistor to ground.
    \item The output also needs a 50 ohm output resistor in series. 
    \item We can use the following circuit from the datasheet: 
    \begin{figure}[h]
        \includegraphics[width=\textwidth]{application_circuit}
    \end{figure}
\end{itemize}

\end{document}
