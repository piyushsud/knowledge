% !TEX root = C:/Users/piyus/knowledge/Project_Specific_Knowledge/public/fm_radio/stages/power/power.tex
\documentclass[12pt, letterpaper]{article}

\usepackage{hyperref}
\usepackage{graphicx}
\graphicspath{ {C:/Users/piyus/knowledge/Project_Specific_Knowledge/public/fm_radio/stages/power/pictures} }

\title{Power Notes}
\author{Piyush Sud}
\date{9/30/2024}
\begin{document}
\maketitle

\pagebreak

\section{High Level Design}

\begin{itemize}
    \item The entire system will be powered by a +15V power supply to generate all the positive voltages and a -15V supply to generate the negative voltages. These will probably be quite noisy, so we will need LDOs for all the logic level supplies. Here's a good supply that produces + or - 15V: \url{https://www.digikey.com/en/products/detail/traco-power/TXLN-035-23M3/13681744}
    \item Let's use the AZ1117ID-ADJTRG1 for our 12V supply. It has a good PSRR of 70 dB @ 120 Hz. This works since Vin = Vout + 2V for the adj supply.
    \item Let's use the AZ1117ID-3.3 and AZ1117ID-5 for our 3.3V and 5V supplies respectively. For those, \(1.5V < Vin - Vout < 10V\), so we can use the 12V supply as Vin so that Vin - Vout = 8.7V for the 3V3 supply and 7V for the 5V supply. 
    \item Let's use the MC79M05CDTRKG to generate the -5V supply from the -15V supply.
    \item Let's use the recommended 10 uF and 22 uF decoupling caps as recommended in the datasheet, and select caps that are low ESR.
\end{itemize}

\section{Calculations}

\begin{itemize}
    \item For the adjustable supply, the resistors needed are given by this formula: VOUT=VREF * (1+R2/R1) + IADJ * R2
    \item The reference voltage is fixed at 1.25V. The IADJ can be ignored since it's very small.
    \item To get a 12V supply from 15V and setting R2 = 2kohms, 2k/R1 = 12/1.25 - 1 so R1 = 232.558139534 ohms. The dropout voltage is 1.3V so powering the 15V supply with 12V is fine.
\end{itemize}

\end{document}