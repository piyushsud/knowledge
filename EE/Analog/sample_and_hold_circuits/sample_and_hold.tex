% !TEX root = C:/Users/piyus/knowledge/EE/Analog/sample_and_hold_circuits/sample_and_hold.tex
\documentclass[12pt, letterpaper]{article}

\usepackage{hyperref}
\usepackage{graphicx}
\graphicspath{ {C:/Users/piyus/knowledge/EE/Analog/sample_and_hold_circuits/pictures} }

\title{Sample and Hold Amplifiers Notes}
\author{Piyush Sud}
\date{10/10/2024}
\begin{document}
\maketitle

\pagebreak

\section{Notes}

\begin{itemize}
    \item The acquisition time is the minimum hold \(=>\) sample time, and the settling time is the minimum sample \(=>\) hold time.
    \item For high frequencies, open-loop configurations are generally used with diode bridge circuits.
    \item Generally, a buffer is required at the output for driving the load in order to prevent charge from leaking from the capacitor.
    \item For closed loop designs, output integrators can be used as the buffer in order to create a virtual ground at the input, which allows the switch to be switched from ground to virtual ground, eliminating leakage current issues.
\end{itemize}


\end{document}